\documentclass{article}
\begin{document}
In this project, we created a Truss Class. A truss is defined as a rigid
figure made up of beams that are joined by frictionless pins.
We are initially given a set of known variables - 
beam orientation, joints orientation, application of external forces, 
and whether or not a joint is connected to a fixed support. We desire 
to attain information regarding the compression force for each beam 
as well the reaction force on the beams due to fixed supports. In order
to solve for these variables, we create a matrix in the form Ax = b, 
where A is our data matrix of known beam coefficient variables separated
into components, x is our unknowns, and b is our vector of external forces.
Using this system of linear equations, we are able to solve for 
compression and reaction forces, gaining valuable knowledge regarding 
the orientation of the truss. This project would be useful to someone 
analyzing the forces necessary for a truss to stay stable. 

In our init class, we initialize all class variables necessary in the 
program. The init method takes as inputs a joints file, a beams file, and
an optional output file. The optional output file is the location where
the program saves the image of the truss generated by our (later discussed)
method, plotgeometry. The init class then stores each file, and initializes
variables and containers used throughout the program. Of these variables 
and containers, it initializes a beams dictionary (which matches beam 
number to the joint indicated for the beam) and a joints dictionary (which
stores the following attributes: x,y coordinates of the joints, Fx and Fy
components of the external force (if any) acting on the joint, and an 
indicator stating whether the joint is attached to a fixed support).
Additionally, it stores a counter which maintains the column index 
needed in the matrix to append our fixed support column, two lists (bxcoeff, 
and bycoeff) which stores our decomposed x and y coordinates for our beam force
coefficients, and a forces list which stores the result vector. The bx/by
coeffs list will be utilized as known values in our data matrix. Lastly, 
the init method invokes the runfunctions method, which runs all the necessary
functions to find the compression forces for each beam and potentially 
plot the truss if desired.  

Next, the getdata method reads beams and joints files for a particular truss. 
It utilizes as an input the beams and joints file initialized in the init 
method. It then parses the files to get data for each beam and joint. 
The method adds tuples of joint indicies as values to a corresponding 
beam number key. Additionally, the method adds tuples of x and y joint 
coordinates, x and y external force components per joint, and an indicator 
for whether or not the joint is attached to a fixed support, as values 
to the cooresponding joint number in the joints dictionary. The method 
also keeps track of the column indicies for the reactionary forces 
in the data matrix. 

Following, we have the getbeamvalues method, which calculates the coefficients
for the components of force for each beam in the linear system. This method 
takes in the beam dictionary, and appends x and y coefficients for each 
beam to their corresponding container. X components are appended to 
bxcoeffs and y components are appended to bycoeffs.

Next, and our meatiest method, is our create matrix. This method creates
a CSR data matrix which contains the linear equations necessary 
to solve our system of equations. The motivation for using a CSR matrix
is to save on storage and to compress the data matrix in a way that 
unnecessary 0 values are not stored. This is a great storage technique 
for sparse matricies, as the computer avoids uneccesary storage 
allocation to the 0s of a larger matrix. We will define our matrix 
as a matrix with beam numbers and reactionary component indicators 
as the columns, and joints are the rows (split up into a row 
for each x and y component). As this method iterates 
through the joints, it locates the beam coefficient for that particular 
component of the joint, and stores it as a value in the sparse matrix.
Additionally, the method adds a 1 for each x and y component of a joint
with a fixed point. This adds the necessary parameters adequately define 
our linear system. While iterating through joints, I create a corresponding 
force vector, which stores the components of the external forces acting 
on the joints. After the iteration, I am left with the proper Ax = b format
to solve the system of linear equations. A is our (sparse) data matrix, x is 
our vector ofunknowns, and b is our force vector. After the iteration, I am able 
to apply the sparse linear package solver from scipy to solve the system 
of linear equations. If the system is over/under determined, meaning 
there are a different number of equations and unknown, I throw 
a runtime error. Additionally, if the matrix is singular, I throw a runtime
error. The final output of this method is a result force vector, which 
is the solution to the linear system and which gives the compression 
forces for each beam. 

Following, we have the optionally invoked plot geometry method. This 
method is only invoked if the user provides an output file for the image. 
The input is the filename for the saved image. The method then utilizes 
the matplotlib module to plot the geometry of the plot based off of the 
orientation for the joints for each beam. The image is then stored 
and saved in the output file. 

Additionally, we have the repr method, which returns the string to be
printed as invoked by main.py. This method uses the class variables as
an input, and returns a string with the desired output. The properly 
formatted beam number and force values appear in a tabular format. 

Lastly, we have the runfunctions method, which is invoked by the init method.
This method runs the necessary functions to obtain beam forces and plot truss
geometry (if applicable). The plot geometry function only runs when the user 
provides the output file for the image. 

And those are all my methods! Thanks for reading :). 

\end{document}
